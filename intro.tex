\section{Introduction}

Timing channel attacks are becoming a major threat as hardware is increasingly 
consolidated and shared by distrusting entities, which traditionally have been
isolated on their own physical machines. For example, in cloud computing, mutually distrusting 
parties own virtual machines on shared hardware. In mobile devices, users often
run downloaded applications that cannot be trusted often handle private personal
information.
While untrusted applications can be sandboxed using
traditional access control mechanisms to limit explicit communications,
timing 
channels can still be used by a malicious program to extract or leak secrets.
%from other co-located applications.
Further, unlike physical side-channel attacks such as power analysis that require
physical proximity, timing channels can be exploited in software by remote
adversaries.

%For example, end users download untrusted 
%applications from the Internet which can then execute on the same hardware as 
%software that will handle the user's sensitive financial data. System on chip 
%platforms tightly integrate hardware designed by directly competing companies.  
%This necessitates hardware sharing among the drivers and proprietary algorithms 
%owned by these distrusting companies. 
%So timing channels 
%vulnerabilities are not only prevalent, but easy to exploit.

%Many of the systems that are vulnerable to timing channels do take security 
%measures to prevent explicit channel attacks. In a cloud platform, the 
%hypervisor performs physical address translation on behalf of the virtual 
%machines to isolate the virtual machines in the physical memory. Hypervisors 
%also use access controls to isolate virtual machines, typically relying on 
%hardware abstractions such as protection rings. However, these security 
%mechanisms are not enough. They do nothing to prevent coresident VMs from 
%leaking secret information through timing channels induced by hardware sharing.

In this paper, we investigate an architecture abstraction, named timing compartments (TCs),
which allows software to explicitly request strong timing isolation among software
entities that share a multi-core processor.
The goal of timing compartments is to remove fine-grained microarchitecture-level timing interference
that cannot be controlled in software and thus enable strong timing isolation that is
comparable to running a software entity on its own processor.
The timing compartment is designed to prevent both unintentional information leaks
through side channels and intentional leaks through covert channels.
%Using the timing compartments, software designers can specify a timing protection
%requirement that is necessary for each application. 
%More importantly, 
When coupled with software-level protection mechanisms to control timing channels 
through coarse-grained events such as scheduling and I/O, timing compartments
can enable comprehensive timing isolation of program or a virtual machine.

The timing compartment is designed so that timing isolation can be
separated from traditional access control. For example, multiple programs or
virtual machines that are isolated from each other using virtual memory may
be grouped into a single timing compartment if they either belong to the 
same trust domain or do not require timing channel protection. 
This separation provides a system the flexibility to only pay overhead for
timing channel protection when it is truly necessary.

To realize the timing compartments, a multi-core processor needs to be
re-designed to control inter-core timing interference in shared hardware
resources that cannot be efficiently controlled in software.
In this paper, we identify all resources that can be shared concurrently among
multiple cores on a typical multi-core processor, namely shared caches,
on-chip interconnect, cache coherence mechanisms, and off-chip memory controllers,
and augment each shared resource with a mechanism
to eliminate timing interference.

This ``full-processor'' protection study revealed new sources timing
channels on a multi-core processor that have not been studied yet.
For example, we found that interfaces between hardware modules and 
a cache coherence mechanism can lead to timing channels, and demonstrate
a covert channel attack using the coherence traffic interference.
The integration effort also showed that protection mechanisms need to
be carefully coordinated in order to avoid unnecessary inefficiencies.
To the best of our knowledge, this paper represents the first study
that implements comprehensive timing channel protection for a full 
multi-core microprocessor and evaluates the overhead.

%Recent studies have investigated mechanisms to prevent
%timing channels in various microarchitecture components, including
%shared caches \cite{icache,newcache,deconstructing,cachegames}, processor pipelines 
%\cite{pipelines}, branch predictors~\cite{branchpred,predictingbranch}, on chip 
%networks~\cite{yaonocs}, and memory controllers~\cite{ushpca14}.
%However, we found that the full timing channel protection for a multi-core
%processor requires more than simply integrating previous protection 
%mechanisms. Our study identified new sources of timing channels at
%interfaces between hardware modules and also found that protection
%mechanisms need to be coordinated together to avoid unnecessary inefficiencies.

%Ascend~\cite{ascend} considers external timing 
%channel attacks, but does not prevent timing channels due to shared resources.  
%Execution leases~\cite{execution_leases} provide an architecture abstraction
%that prevents external timing channels by bounding the execution time of a code 
%segment, but Execution leases do not prevent the internal timing channels 
%caused by shared hardware. GLIFT \cite{citation_needed} can be used to identify 
%timing channels, but does not prevent them. Further, the area, power, and 
%performance overheads are prohibitively large.

%Unlike previous work, timing compartments eliminate all internal timing 
%channels in a conventional microarchitecture. When combined with standard 
%access controls, timing compartments achieve the security of running mutually 
%distrusting software on separate hardware with some of the performance benefits 
%of sharing hardware. To the best of our knowledge, this is the first 
%architecture technique to reach this goal. This work is also the first to 
%experimentally evaluate the cost of applying timing channel protection to every 
%vulnerable component. Further, we show that integrating these protection 
%mechanisms to form a complete system with minimal performance overheads is 
%nontrivial and requires careful coordination. In the process of designing an 
%integrated timing channel protection system we identified two novel 
%microarchitectural timing channels. In addition to these contributions we 
%extend the taxonomy of timing channels and show how this taxonomy can be used 
%to identify techniques that can eliminate timing channels in any additional 
%components (e.g. accelerators).

The simulation results suggest that the performance overhead of supporting
timing compartments is relatively low, especially when the number of timing
compartments that simultaneously run is small.
%for the SPEC benchmarks. 
Even though
the timing compartments require shared resources to be statically 
allocated, the overall impact on the performance is limited by the fact
that the protection only affects infrequent cache misses and that modern
processors include large caches. 

The following summarizes the main contributions.

\begin{itemize}
\item The paper introduces a new abstraction that enables software to
explicitly remove microarchitecture-level timing interference on a multi-core.
%The timing compartment enables new
%applications that require strong timing isolation of software modules.
\item The paper identifies new timing channels on a multi-core processor,
including the one through cache coherence, and presents
comprehensive protection mechanisms for a full multi-core processor.
\item The paper shows how protection mechanisms need to be coordinated to
reduce the overhead.% of timing compartments.
\item The paper evaluates the performance overhead of the full-processor
timing channel protection, and shows that the overhead can be reasonable.
\end{itemize}

The rest of the paper is organized as follows.
Section 2 introduces timing compartments and 
presents example applications that can be enabled by strong timing isolation.
Section 3 identifies the sources of timing channels in a multi-core processor, and
describes protection mechanisms to eliminate them. 
Section 4 presents how the hardware protection mechanisms based on time-division
multiplexing need to be coordinated to reduce overhead.
Section 5 evaluates the proposed architecture. Section 6 discusses related
work, and Section 7 concludes the paper.
