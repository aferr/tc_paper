\section{Introduction}
Though once considered only a problem for small embeded devices like 
smartcards, microarchitectural timing channel attacks have become a severe 
threat to general purpose computing devices in diverse range of application 
domains.Computer hardware is increasingly shared among distrusting parties, and 
since a large subset of microarchitectural timing channel attacks are due to 
interference in shared resources the problem has grown.  End users download 
untrusted applications from the internet which can then execute on the same 
hardware as tax software that will handle the user's senitive financial data.
System on chip platforms tightly integrate hardware designed by directly 
competing companies necessitating sharing among the drivers and proprietary 
algorithms owned by these distrusting companies.
In cloud computing, mutually distrusting parties own virutal machines on shared 
hardware. The hypervisor performs physical address translation on behalf of the 
virtual machines to isolate the virtual machines in the physical memory.  
Hypervisors also use access controls to isolate virtual machines, typically 
relying on hardware abstractions such as protection rings. However, these 
security mechanisms are not enough. They do nothing to prevent coresident VMs 
from leaking secret information through timing channels induced by hardware 
sharing.

We present Timing Compartments, an architecture abstraction to 
provide timing channel isolation among software modules that share hardware.
Timing Compartments provide an interface to software designers that allows them 
to specify a timing channel protection policy that, when coupled with 
conventional access controls, provides total software isolation. To implement 
Timing Compartments, we present a full microprocessor design that eliminates 
microarchitectural timing channels in every shared hardware resource, and we 
evaluate its overhead.

Other work has focused on eliminating timing channel attacks in a specific 
microarchitectural component such as shared caches 
\cite{icache,newcache,deconstructing,cachegames}, processor pipelines 
\cite{pipelines}, branch predictors~\cite{branchpred,predictingbranch}, 
on chip networks~\cite{yaonocs}, and memory controllers~\cite{ushpca14}.
