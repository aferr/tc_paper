\section{Evaluation}
We evaluated the timing compartments architecture using the gem5 architectural 
simulator~\cite{gem5} interfaced with the DRAMSim2~\cite{DRAMSim2} memory and 
memory controller simulator. Our experiments use multiprogram workloads 
comprised of groups of SPEC2006 benchmarks compiled for the ARM ISA. 

The 2GHz out-of-order cores use the gem5 ``O3`` model. 
Each core has private 32KB L1 instruction and data caches, and private 256KB L2 
cache. The cores share a 4MB L3 cache. We derived cache configuration 
parameters from the Intel Xeon E3-1220L, which is a two core architecture used 
by Amazon EC2. In DRAMSim2, we simulate a 667MHz 2GB DDR3 memory. The 
interconnects in the simulator have a 1GHz clock. Each experiment is 
fastforwarded for 1 billion instructions, and run for 100 million 
instructions.

\begin{table}
    \caption{Simulator configuration parameters.}
    \centering
    \begin{tabular}{|l|l|l|r|}
        \hline
        \multicolumn{3}{|l|}{gem5 core model} & ``O3''        \\\hline
        \multicolumn{3}{|l|}{CPU Clock}    & 2GHz             \\\hline
        \hline
        \multicolumn{2}{|l|}{Memory}             & 2GB    & 667MHz  \\\hline
        \hline
        \multicolumn{3}{|l|}{Network Clock}      & 1GHz \\\hline
        \hline
        L1d / L1i  & 32kB   & 2-way  & 2 cycles\\\hline
        L2         & 256kB  & 8-way  & 7 cycles \\\hline
        L3         & 4MB    & 16-way & 17 cycles  \\\hline
    \end{tabular}
    \label{tab:config}
\end{table}

\subsection{Security Evaluation}
To experimentally evaluate the security of the temporal partitioning 
architecture, we use a two-core system with separate timing compartments, TC1 
and TC2,
allocated to each core and a policy that disallows timing channel leakage from
either core. If the number of cycles required to reach 100 million instructions 
for a particular benchmark running in TC1 depends which benchmark is running in 
TC2, there must be interference which can cause timing channel leakage. For 
each benchmark, we compare its execution time while sharing hardware with each 
other SPEC benchmark. 
%Fingers crossed!
During our tests we observed no cycle count differences by changing the 
benchmark running in the other timing compartment.

\subsection{Performance Evaluation}
