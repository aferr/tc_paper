\title{
    Timing Compartments: An Architecture Abstraction for Timing Isolation
}

\ifanonymized{
    \author{}
}{
    \author{
    Andrew Ferraiuolo, Yao Wang, and G. Edward Suh\thanks{The first two authors 
    contributed equally to this work.}\\
    Cornell University\\
    Ithaca, NY 14850, USA\\
    \{af433,yw438,gs272\}@cornell.edu
    }
}


\date{}
\maketitle

\thispagestyle{empty}

\begin{abstract}
    This paper presents timing compartments, a hardware architecture primitive 
    that eliminates microarchitectural timing channels between groups of 
    processes or VMs. Timing channels violate the boundaries between these 
    software system entities despite conventional security techniques such as 
    access control and virtual memory. Unlike other forms of covert/side channels such 
    as power consumption, timing channels can be exploited in software without 
    physical access to the device. 
%    When coupled with conventional security 
%    techniques, timing compartments enable distrusting entities to share 
%    hardware with a level of assurance that is comparable to executing
%    on separate hardware. By separating timing channel control from control for 
%    explicit communication channels, timing compartments afford the flexibility 
%    to pay for timing channel protection only when necessary. 
    To realize timing 
    compartments, we design and experimentally evaluate a full multi-core 
    processor that eliminates timing channels for critical shared hardware 
    components. 
    In particular, we identified new sources of timing channels including
    cache coherence mechanisms and module interfaces, and provide solutions for them.
    The experimental results suggest that the overheads of
    timing compartments can be rather low in modern microprocessors, especially 
    when the number of timing compartments is small.

\end{abstract}
