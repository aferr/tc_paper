\title{
\vspace{-0.1in}
    Timing Compartments: An Architecture Primitive for Complete Software 
    Isolation
}

\ifanonymized{
    \author{}
}{
    \author{
    Andrew Ferraiuolo, Yao Wang, and G. Edward Suh\thanks{The first two authors 
    contributed equally to this work.}\\
    Cornell University\\
    Ithaca, NY 14850, USA\\
    \{af433,yw438,gs272\}@cornell.edu
    }
}


\date{}
\maketitle

\thispagestyle{empty}

\begin{abstract}
    This paper presents timing compartments, a hardware architecture primitive 
    that completely eliminates microarchitectural timing channels between 
    groups of software entities (e.g. processes or virtual machines). 
    %specified by the system owner. 
    When coupled with conventional techniques to isolate software such as
    virtual memory, timing compartments can enables removin all communication channels
    that can be exploited in software.
    This capability enables distrusting entities to share hardware with
    a level of assurance that is comparable to executing on 
    separate hardware.
    By separating timing channel control from control for explicit communication
    channels, timing compartments afford the flexibility to pay for timing 
    channel protection only when necessary. To realize timing compartments, we 
    design and experimentally evaluate a full multi-core processor that eliminates 
    timing channels for all shared hardware components.
    The experimental results suggest that the overhead of supporting 
    timing compartments can be rather low in modern microprocessors, especially when
    the number of timing compartments is small.

\end{abstract}
